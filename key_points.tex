\documentclass[]{article}
\usepackage{lmodern}
\usepackage{amssymb,amsmath}
\usepackage{ifxetex,ifluatex}
\usepackage{fixltx2e} % provides \textsubscript
\ifnum 0\ifxetex 1\fi\ifluatex 1\fi=0 % if pdftex
  \usepackage[T1]{fontenc}
  \usepackage[utf8]{inputenc}
\else % if luatex or xelatex
  \ifxetex
    \usepackage{mathspec}
  \else
    \usepackage{fontspec}
  \fi
  \defaultfontfeatures{Ligatures=TeX,Scale=MatchLowercase}
\fi
% use upquote if available, for straight quotes in verbatim environments
\IfFileExists{upquote.sty}{\usepackage{upquote}}{}
% use microtype if available
\IfFileExists{microtype.sty}{%
\usepackage{microtype}
\UseMicrotypeSet[protrusion]{basicmath} % disable protrusion for tt fonts
}{}
\usepackage[margin=1in]{geometry}
\usepackage{hyperref}
\hypersetup{unicode=true,
            pdftitle={key\_points},
            pdfauthor={Thiyanga Talagala},
            pdfborder={0 0 0},
            breaklinks=true}
\urlstyle{same}  % don't use monospace font for urls
\usepackage{graphicx,grffile}
\makeatletter
\def\maxwidth{\ifdim\Gin@nat@width>\linewidth\linewidth\else\Gin@nat@width\fi}
\def\maxheight{\ifdim\Gin@nat@height>\textheight\textheight\else\Gin@nat@height\fi}
\makeatother
% Scale images if necessary, so that they will not overflow the page
% margins by default, and it is still possible to overwrite the defaults
% using explicit options in \includegraphics[width, height, ...]{}
\setkeys{Gin}{width=\maxwidth,height=\maxheight,keepaspectratio}
\IfFileExists{parskip.sty}{%
\usepackage{parskip}
}{% else
\setlength{\parindent}{0pt}
\setlength{\parskip}{6pt plus 2pt minus 1pt}
}
\setlength{\emergencystretch}{3em}  % prevent overfull lines
\providecommand{\tightlist}{%
  \setlength{\itemsep}{0pt}\setlength{\parskip}{0pt}}
\setcounter{secnumdepth}{0}
% Redefines (sub)paragraphs to behave more like sections
\ifx\paragraph\undefined\else
\let\oldparagraph\paragraph
\renewcommand{\paragraph}[1]{\oldparagraph{#1}\mbox{}}
\fi
\ifx\subparagraph\undefined\else
\let\oldsubparagraph\subparagraph
\renewcommand{\subparagraph}[1]{\oldsubparagraph{#1}\mbox{}}
\fi

%%% Use protect on footnotes to avoid problems with footnotes in titles
\let\rmarkdownfootnote\footnote%
\def\footnote{\protect\rmarkdownfootnote}

%%% Change title format to be more compact
\usepackage{titling}

% Create subtitle command for use in maketitle
\providecommand{\subtitle}[1]{
  \posttitle{
    \begin{center}\large#1\end{center}
    }
}

\setlength{\droptitle}{-2em}

  \title{key\_points}
    \pretitle{\vspace{\droptitle}\centering\huge}
  \posttitle{\par}
    \author{Thiyanga Talagala}
    \preauthor{\centering\large\emph}
  \postauthor{\par}
      \predate{\centering\large\emph}
  \postdate{\par}
    \date{March 27, 2019}


\begin{document}
\maketitle

``How to choose the best techniqueor combination of techniques to help
solve your particular forecasting dilemma'' from Manager's guide to
forecasting by David M. Georgoff and Robert G. Murdick. A large and
fastgrowing body of research deals with the development, refinement, and
evaluation of forecast techniques

Selecting a most appropriate forecast model or a combination of models
to use in forecasting is a challenging task. Such problems typically
lack algebraic expressions, it is not possible to calculate derivative
information, and the problem may exhibit uncertainty or noise. Expert
knowledge is required. Even with the necessary knowledge and skills,
success is not quaranteed.

since prior expert knowledge is often expensive, not always readily
available, and subject to bias and personal preferences, metalearning
can serve as a promising complement to this form of advice through the
automatic accumulation of experience based on the performance of
multiple applications of a learning system.

Reid (1972) was among the first to argue that the relative accuracy of
forecasting methods changes according to the properties of the time
series (Evidence for the Selection of Forecasting Methods by NIGEL
MEADE).

\begin{itemize}
\tightlist
\item
  Comparison with the previous frameworks we have developed
\end{itemize}

\begin{enumerate}
\def\labelenumi{\arabic{enumi}.}
\item
  Feature-based FORecast Model Selection (FFORMS): poses the problem as
  a classification problem using the random forest algorithm. The model
  output is the ``best'' forecasting method.
\item
  Feature-based FORecast Model Averaging (FFORMA): poses the problem as
  a average forecast error minimization problem using the extreme
  gradient boosting approach. The output is a vector of weights assigns
  to each forecast-model.
\item
  \textcolor{red}{Feature-based FORecast Model Performance Prediction (FFORMPP)}
  {[}tentative name{]}: poses the problem as a multivariate regression
  problem in which we try to predict forecast-error of each method. The
  main difference compared to our previous approaches is the output, Y,
  is vector, which means we predict the performances of several
  forecasting methods simultaneously by taking the correlation structure
  into account. These statistics are used as explanatory variables in
  predicting the relative performance of the nine methods using a set of
  simulated time series with known properties. These results are
  evaluated on observed data sets, the M-Competition data and Fildes
  Telecommunications data. The general conclusion is that the summary
  statistics can be used to select a good forecasting method (or set of
  methods) but not necessarily the best.
\end{enumerate}

Rather than mapping a dataset to a single predictive model, one may also
produce a ranking over a set of different models. One can argue that
such rankings are more flexible and informative for users. In a
practical scenario, users should not be limited to a single kind of
advice; this is important if the suggested final model turns
unsatisfactory. Rankings provide alternative solutions to users who may
wish to incorporate their own expertise or any other criterion (e.g.,
financial constraints) on their decision-making process.

Enables us to select one or more strategies that seem effective given
the characteristics of the dataset under analysis.

\begin{itemize}
\tightlist
\item
  simulate time series using MAR models
\end{itemize}

quality of the predictions normally improves with an increasing number
of scenarios or examples.

It has been widely accepted that gathering additional training data
typically leads to an improvement in performance of predictive
models(paper: Cross-domain Meta-learning for Time-series Forecasting).

Additionally, in a real-world environment it is common for frequent
changes to the time series portfolio as new relevant time series are
added to the system and other nolonger relevant time series are removed
from the system. Randomly generated test instances lack diversity and
rarely resemble real-world instances.

\begin{itemize}
\item
  use additional data from other domains, but doesn't guarantee that it
  will cover the entire space
\item
  This paper in this series
\item
  The forecasting chart can help the manager select the best combination
  of techniques.
\end{itemize}

We first use random forest algorithms to predict the forecast-model that
is expected to perform best on a given time series. This is known as the
algorithm selection problem, and can be expressed formally using Rice's
framework for algorithm selection. We then applied xgboost algorithm to


\end{document}
